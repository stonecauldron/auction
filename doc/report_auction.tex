\documentclass[11pt]{article}

\usepackage{amsmath}
\usepackage{textcomp}
\usepackage[top=0.8in, bottom=0.8in, left=0.8in, right=0.8in]{geometry}
% add other packages here

% put your group number and names in the author field
\title{\bf Exercise 5: An Auctioning Agent for the Pickup and Delivery Problem}
\author{Group 39: Hugo Bonnome, Pedro Amorim}

\begin{document}
\maketitle

\section{Bidding strategy}
% describe in details your bidding strategy. Also, focus on answering the
% following questions:

% - do you consider the probability distribution of the tasks in defining your
% strategy? How do you speculate about the future tasks that might be auctions?

% - how do you use the feedback from the previous auctions to derive information
% about the other competitors?

% - how do you combine all the information from the probability distribution of
% the tasks, the history and the planner to compute bids?
\subsection{General outline}
We use these general assumptions to build our strategy:
\begin{itemize}
\item The costliest actions are when vehicles travel with empty trunks. Having
  parcels to deliver while going to a specific destination offsets the cost of
  travel.
\item There are no time constraints to deliver tasks. As such there is no
  penalty in accepting a large number of tasks and delaying their deliveries and
  pickup times.
\item Since there are only two agents playing the competition can be assimilated
  to a zero sum game; i.e. only the relative difference with the profit from the
  adversary counts, not the absolute value.
\item There is a lower bound on the number of tasks that is equal to 10.
\end{itemize}

With these assumptions in mind we constructed our strategy as follows. Since
there is going to be at least 10 tasks into play, a task that might seem
disadvantageous at the beginning might become profitable if it is followed by
other deliveries and pickups. As such it is advantageous to fill the trunks of
the cars at the beginning. To ensure this our bidding strategy starts very
aggressively, with the possibility of bidding lower than our own current
marginal cost with the hope that this loss will be minimized when the number of
accepted tasks reaches a certain threshold.This also has the effect of
preventing the adversary agent from winning tasks early and ensures that we will
keep a head start in the profit differences.

\subsection{Implementation details}
% Explain in detail how we compute bids
% First explain how we estimate the marginal cost of the adversary
% Then how the bids are computed

\section{Results}
% in this section, you describe several results from the experiments with your auctioning agent

\subsection{Experiment 1: Comparisons with dummy agents}
% in this experiment you observe how the results depends on the number of tasks auctioned. You compare with some dummy agents and potentially several versions of your agent (with different internal parameter values). 

\subsubsection{Setting}
% you describe how you perform the experiment, the environment and description of the agents you compare with

\subsubsection{Observations}
% you describe the experimental results and the conclusions you inferred from these results

\vdots

\subsection{Experiment n}
% other experiments you would like to present (for example, varying the internal parameter values)

\subsubsection{Setting}

\subsubsection{Observations}

\end{document}
